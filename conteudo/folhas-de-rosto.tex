%%%%%%%%%%%%%%%%%%%%%%%%%%% CAPA E FOLHAS DE ROSTO %%%%%%%%%%%%%%%%%%%%%%%%%%%%%

% Embora as páginas iniciais *pareçam* não ter numeração, a numeração existe,
% só não é impressa. O comando \mainmatter (mais abaixo) reinicia a contagem
% de páginas e elas passam a ser impressas. Isso significa que existem duas
% páginas com o número "1": a capa e a página do primeiro capítulo. O pacote
% hyperref não lida bem com essa situação. Assim, vamos desabilitar hyperlinks
% para números de páginas no início do documento e reabilitar mais adiante.
\hypersetup{pageanchor=false}

% A capa; o parâmetro pode ser "port" ou "eng" para definir a língua
\capaime[port]
%\capaime[eng]

% Se você não quiser usar a capa padrão, você pode criar uma outra
% capa manualmente ou em um programa diferente. No segundo caso, é só
% importar a capa como uma página adicional usando o pacote pdfpages.
%\includepdf{./arquivo_da_capa.pdf}

% A página de rosto da versão para depósito (ou seja, a versão final
% antes da defesa) deve ser diferente da página de rosto da versão
% definitiva (ou seja, a versão final após a incorporação das sugestões
% da banca). Os parâmetros podem ser "port/eng" para a língua e
% "provisoria/definitiva" para o tipo de página de rosto.
%\pagrostoime[port]{definitiva}
\pagrostoime[port]{provisoria}
%\pagrostoime[eng]{definitiva}
%\pagrostoime[eng]{provisoria}

%%%%%%%%%%%%%%%%%%%% DEDICATÓRIA, RESUMO, AGRADECIMENTOS %%%%%%%%%%%%%%%%%%%%%%%

% A definição deste ambiente está no pacote imeusp; se você não
% carregar esse pacote, precisa cuidar desta página manualmente.
%%%\begin{dedicatoria}
%%%Esta seção é opcional e fica numa página separada; ela pode ser usada para
%%%uma dedicatória ou epígrafe.
%%%\end{dedicatoria}

% Após a capa e as páginas de rosto, começamos a numerar as páginas; com isso,
% podemos também reabilitar links para números de páginas no pacote hyperref.
% Isso porque, embora contagem de páginas aqui começe em 1 e no primeiro
% capítulo também, o fato de uma numeração usar algarismos romanos e a outra
% algarismos arábicos é suficiente para evitar problemas.
\pagenumbering{roman}
\hypersetup{pageanchor=true}

% Agradecimentos:
% Se o candidato não quer fazer agradecimentos, deve simplesmente eliminar
% esta página. A epígrafe, obviamente, é opcional; é possível colocar
% epígrafes em todos os capítulos. O comando "\chapter*" faz esta seção
% não ser incluída no sumário.
\chapter*{Agradecimentos}
%\epigrafe{Do. Or do not. There is no try.}{Mestre Yoda}

Várias pessoas contribuiram para que fosse possível a conclusão deste trabalho, dedico esse feito a todos e meu muito obrigado por me acompanharem nessa jornada.
Gostaria de agradecer imensamente o apoio incondicional da minha família, sem ela não teria sido possível a minha permanência em São Paulo ao longo dos últimos anos, além do amor e carinho que sempre foi
presente em toda a minha vida.
Agradeço também o apoio de todos os amigos que tornaram essa mudança total de vida muito agradável e inesquecível.
Sou muito grato também a todos os colegas de laboratório que me ajudaram a crescer tecnicamente e academicamente no decorrer dos últimos anos.
E por fim, deixo aqui o meu reconhecimento a todos os professores que fizeram parte da minha jornada acadêmica, principalmente aos que tiverem paciência e me orientaram nessa empreitada: Prof. Dr. Paulo 
Meirelles durante a graduação e o Prof. Dr. Fabio Kon durante este mestrado.
Esta conquista não é apenas minha e sim de todos vocês.
Obrigado.

% O resumo é obrigatório, em português e inglês. Este comando também gera
% automaticamente a referência para o próprio documento, conforme as normas
% sugeridas da USP
\begin{resumo}{port}
Cidades ao redor do mundo enfrentam diversos desafios para proporcionar uma boa qualidade de vida aos seus cidadãos.
Sistemas de software vêm sendo desenvolvidos com objetivo de melhorar os serviços e otimizar o uso da infraestrutura da cidade.
Desenvolver ambientes de experimentação para esses sistemas na escala de grandes cidades ainda é um desafio, devido ao alto custo e problemas de infraestrutura.
Por sua vez, a simulação é um mecanismo que vem sendo utilizado na realização de experimentos em diversas áreas do conhecimento.
O objetivo deste trabalho é prover um mecanismo para construção de um ambiente de experimentação de larga escala e interativo para plataformas de Cidades Inteligentes através de simulação.
Uma arquitetura de software foi desenvolvida visando permitir a integração de plataformas e simuladores de Cidades Inteligentes.
Dois estudos de caso demostraram a viabilidade da solução, integrando o simulador InterSCSimulator e a plataforma InterSCity.
Foram apresentados detalhes de como implementar a arquitetura proposta, além da execução de experimentos na escala da cidade de São Paulo.
A solução foi efetiva, tendo em vista que, foi possível realizar experimentos de larga escala através de simulação por meio da implementação da arquitetura apresentada.
\end{resumo}

% O resumo é obrigatório, em português e inglês. Este comando também gera
% automaticamente a referência para o próprio documento, conforme as normas
% sugeridas da USP
\begin{resumo}{eng}
Cities around the world face a number of challenges to provide a good quality of life for their citizens.
Software systems have been developed with the aim of improving services and optimizing the use of the city's infrastructure.
Developing experimentation environments for these systems in the large cities scale is still a challenge due to the high cost and infrastructure problems.
In turn, the simulation is a mechanism that has been used to enable experiments in several areas of knowledge.
The objective of this work is to provide a mechanism for developing a large scale and interactive experimentation environment for Smart Cities platforms through simulation.
A software architecture was designed to allow the integration of platforms and simulators of Smart Cities.
Two case studies demonstrated the viability of the solution, integrating the InterSCSimulator simulator and the InterSCity platform.
We presented details of how to implement the proposed architecture, besides the execution of experiments in the scale of the city of São Paulo.
The solution was effective, considering that, it was possible to carry out large scale experiments using simulation through the implementation of the presented architecture.
\end{resumo}

% Como as listas que se seguem podem não incluir uma quebra de página
% obrigatória, inserimos uma quebra manualmente aqui.
\makeatletter
\if@openright\cleardoublepage\else\clearpage\fi
\makeatother

%%%%%%%%%%%%%%%%%%%%%%%%%%% LISTAS DE FIGURAS ETC. %%%%%%%%%%%%%%%%%%%%%%%%%%%%%

% Todas as listas são opcionais; Usando "\chapter*" elas não são incluídas
% no sumário. As listas geradas automaticamente também não são incluídas
% por conta das opções "notlot" e "notlof" que usamos mais acima.

% Normalmente, "\chapter*" faz o novo capítulo iniciar em uma nova página, e as
% listas geradas automaticamente também por padrão ficam em páginas separadas.
% Como cada uma destas listas é muito curta, não faz muito sentido fazer isso
% aqui, então usamos este comando para desabilitar essas quebras de página.
% Se você preferir, comente as linhas com esse comando e des-comente as linhas
% sem ele para criar as listas em páginas separadas. Observe que você também
% pode inserir quebras de página manualmente (com \clearpage, veja o exemplo
% mais abaixo).
\newcommand\disablenewpage[1]{{\let\clearpage\par\let\cleardoublepage\par #1}}

% Nestas listas, é melhor usar "raggedbottom" (veja basics.tex). Colocamos
% a opção correspondente e as listas dentro de um par de chaves para ativar
% raggedbottom apenas temporariamente.
{
\raggedbottom

%%%%% Listas criadas manualmente

%\chapter*{Lista de Abreviaturas}
%%%\disablenewpage{\chapter*{Lista de Abreviaturas}}
%%%
%%%\begin{tabular}{rl}
%%%         CFT         & Transformada contínua de Fourier (\emph{Continuous Fourier Transform})\\
%%%         DFT         & Transformada discreta de Fourier (\emph{Discrete Fourier Transform})\\
%%%        EIIP         & Potencial de interação elétron-íon (\emph{Electron-Ion Interaction Potentials})\\
%%%        STFT         & Transformada de Fourier de tempo reduzido (\emph{Short-Time Fourier Transform})\\
%%%	ABNT         & Associação Brasileira de Normas Técnicas\\
%%%	URL          & Localizador Uniforme de Recursos (\emph{Uniform Resource Locator})\\
%%%	IME          & Instituto de Matemática e Estatística\\
%%%	USP          & Universidade de São Paulo
%%%\end{tabular}

%\chapter*{Lista de Símbolos}
%%%\disablenewpage{\chapter*{Lista de Símbolos}}
%%%
%%%\begin{tabular}{rl}
%%%        $\omega$    & Frequência angular\\
%%%        $\psi$      & Função de análise \emph{wavelet}\\
%%%        $\Psi$      & Transformada de Fourier de $\psi$\\
%%%\end{tabular}

% Quebra de página manual
\clearpage

%%%%% Listas criadas automaticamente

%\listoffigures
\disablenewpage{\listoffigures}

%\listoftables
\disablenewpage{\listoftables}

% Esta lista é criada "automaticamente" pela package float quando
% definimos o novo tipo de float "program" (em utils.tex)
%\listof{program}{\programlistname}
%%%\disablenewpage{\listof{program}{\programlistname}}

% Sumário (obrigatório)
\tableofcontents

} % Final de "raggedbottom"

% Referências indiretas ("x", veja "y") para o índice remissivo (opcionais,
% pois o índice é opcional). É comum colocar esses itens no final do documento,
% junto com o comando \printindex, mas em alguns casos isso torna necessário
% executar texindy (ou makeindex) mais de uma vez, então colocar aqui é melhor.
%%%\index{Inglês|see{Língua estrangeira}}
%%%\index{Figuras|see{Floats}}
%%%\index{Tabelas|see{Floats}}
%%%\index{Código-fonte|see{Floats}}
%%%\index{Subcaptions|see{Subfiguras}}
%%%\index{Sublegendas|see{Subfiguras}}
%%%\index{Equações|see{Modo Matemático}}
%%%\index{Fórmulas|see{Modo Matemático}}
%%%\index{Rodapé, notas|see{Notas de rodapé}}
%%%\index{Captions|see{Legendas}}
%%%\index{Versão original|see{Tese/Dissertação, versões}}
%%%\index{Versão corrigida|see{Tese/Dissertação, versões}}
%%%\index{Palavras estrangeiras|see{Língua estrangeira}}
%%%\index{Floats!Algoritmo|see{Floats, Ordem}}
