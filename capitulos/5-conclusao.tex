\chapter{Conclusão e Trabalhos Futuros}
\label{cap:conclusao}

\section{Conclusão}

%Proposta de integração de plataforma e simulador para prover o ambiente de
%experimentação
%
%Viabilidade da proposta através de dois estudos de caso
%
%Ambiente de experimentação emulado é uma opção diante da necessidade de
%avaliaçãoes dessas plataformas em larga-escala
%
%Artigos publicados/a serem publicados

A proposta de arquitetura para criação de um ambiente simulado de experimentação para plataformas de Cidades Inteligentes se apresentou como uma solução viável, tendo em vista que fomos capazes de
realizar experimentos de diferentes naturezas utilizando implementações feitas seguindo a solução apresentada.
Durante a pesquisa, vimos que a utilização de simulação é uma saída para a realização de experimentos na escala de grandes cidades e sem a necessidade de grandes investimentos em \textit{testbeds} reais.
Além de permitir a realização de experiências com tecnologias ainda não difundidas ou até mesmo não existentes em algumas cidades, como o uso de Placas de Mensagens Variadas (PMVs) notificando motoristas
em tempo real sobre problemas no trânsito na cidade de São Paulo.

Concluímos também que a tarefa de criar um ambiente de experimentação envolvendo a integração de ferramentas não é trivial, corroborando com a literatura.
Apesar de ambas as ferramentas utilizadas para implementação da arquitetura proposta serem desenvolvidas pelo nosso grupo de pesquisa, encontramos diversos entraves durante o processo.
Esses problemas nos mostraram a importância da utilização de ferramentas de código aberto, pois muitas vezes o componente de integração, introduzido pela nossa proposta de solução, não é capaz de
solucionar problemas inerentes as ferramentas.
Problemas esses que, em geral, só são evidenciados em cenários similares ao contexto enfrentado por metrópoles, demonstrando a importância da realização desse tipo de experimento durante o ciclo de
desenvolvimento de plataformas para Cidades Inteligentes.

Ademais, vimos que apenas a existência de um ambiente simulado que permita a realização de experimentos em escalas maiores não soluciona todos os problemas, a definição precisa do cenário de simulação
através de dados reais de cidades tem suma relevância.
Como apresentado na Seção \ref{sec:smart_traffic}, os resultados obtidos no segundo estudo de caso não foram melhores devido ao não conhecimento detalhado dos dados de entrada utilizados.
Outro ponto importante na realização de experimentos é a automatização desse processo, contribuindo com trabalhos que possam ser derivados através da reprodução dos mesmos, fazendo uso de ferramentas
e técnicas de infraestrutura como código (IaC).

Por fim, utilizando o ambiente de experimentação simulado desenvolvido neste trabalho, publicamos o artigo \textit{Design and evaluation of a scalable smart city software platform with
large-scale simulations}\footnote{https://www.sciencedirect.com/science/article/pii/S0167739X18307301} juntamente com colegas do nosso grupo de pesquisa, na revista \textit{Future Generation of
Computer Systems} edição 93 que será publicada em abril de 2019.  
Essa publicação em uma revista bem conceituada foi de suma importância para demostrarmos que a abordagem apresentada neste trabalho é factível e inovadora na área de Cidades Inteligentes.


\section{Limitações do trabalho}

%Arquitetura depende das ferramentas suportarem os protocolos apresentados

A primeira limitação deste trabalho é o fato de a proposta de solução apresentada não ter sido validada com outras ferramentas além do InterSCSimulator e da InterSCity.
Essa validação seria interessante para afirmarmos que a solução proposta tem um âmbito mais genérico.
Ainda sobre a arquitetura de integração, não tivemos a oportunidade de implementar um módulo de tradução semântica, já que ambas as ferramentas tinham conceitos similares implementados.
Sabemos que o mesmo é de suma importância em casos onde as ferramentas não são desenvolvidas pelo mesmo grupo de pessoas, por exemplo.

Além disso, os estudos de casos apresentam algumas limitações, principalmente o que contempla o cenário de tráfego de carros inteligente.
Como já foi dito, as entradas para a simulação deveriam ter sido melhores analisadas previamente, o que acabou nos levando a obter resultados menos expressivos.
Todavia, acreditamos que apesar dessa limitação do experimento, o estudo de caso cumpriu o seu papel que era verificar a viabilidade de execução de experimentos fazendo uso de comandos de atuação no
ambiente simulado.

\section{Trabalhos Futuros}

%Integração de novos cenários de cidades inteligentes seguindo a proposta de
%solução apresentada (InterSCity + InterSCSimulator)
%
%Implementar a solução proposta utilizando outras ferramentas (plataforma e
%simulador/emulador)
%
%Implementar a solução proposta utilizando outros protocolos de rede (MQTT, web
%sockets, ...)

Para a continuidade deste trabalho seria interessante a utilização de diferentes ferramentas para a implementação da solução proposta.
Ao utilizar uma plataforma e um simulador diferente, podemos validar a arquitetura apresentada em um contexto diferente, principalmente com o que diz respeito ao componente de integração, que provavelmente
necessitará de um módulo de tradução semântica e mais complexidade embutida em sua implementação.

Nesse sentido, também seria interessante adicionar novos cenários de Cidades Inteligentes no ambiente já integrado neste trabalho, com o InterSCSimulator e a InterSCity, e realizar novos experimentos.
Como foi visto no decorrer do trabalho, novos cenários trazem novos desafios e requisitos fundamentais que podem precisar ser implementados ou evoluídos dentro das ferramentas mencionadas.
Quanto mais cenários de Cidades Inteligentes ao ambiente de experimentação InterSCSimulator + InterSCity nós adicionarmos, mais interessante se tornará a nossa solução, além de estarmos avaliando a
plataforma InterSCity constantemente em condições realistas.

A utilização de outros protocolos de rede para realizar a comunicação entre o simulador e a plataforma também seria algo interessante.
Neste trabalho utilizamos apenas os protocolos HTTP e AMQP nas implementações dos dois cenários de Cidades Inteligentes estudados, protocolos como o MQTT\footnote{http://mqtt.org/} vem sendo amplamente
utilizados para a comunicação com dispositivos IoT, por exemplo.
