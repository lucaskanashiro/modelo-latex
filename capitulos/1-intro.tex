\chapter{Introdução}
\label{cap:introducao}

% Contextualização: cidades inteligentes

A utilização de Tecnologia da Informação e Comunicação (TIC) para solucionar os problemas das cidades está surgindo como uma boa estratégia para mitigar os problemas que são agravados pelo rápido crescimento da população urbana~\cite{chourabi_2012}.
Essa abordagem visa otimizar os serviços e uso de recursos da cidade, assim como melhorar a qualidade de vida dos cidadãos~\cite{santana_2016}.

Para que plataformas de Cidades Inteligentes atendam as expectativas de governos e cidadãos na melhoria da qualidade de vida nas cidades, elas precisam ser capazes de lidar com todas as adversidades e
complexidade presentes nas cidades modernas.
Devem ser flexíveis e capazes de escalar para atender as diferentes demandas no decorrer do dia, além de conseguir lidar com múltiplos aspectos que tornam o contexto de Cidades Inteligentes tão
peculiar.
Nas cidades, tudo é conectado, um aspecto influencia e é influenciado por diversos outros.

Em uma grande metrópole como São Paulo, milhões de recursos (ônibus, carros, hospitais, estacionamentos, escolas, etc.) deverão ser monitorados e gerenciados em tempo real.
Dispositivos de Internet das Coisas (\textit{Internet of Things - IoT}) poderão enviar e receber dados a todo momento, concorrentemente, com o objetivo de obter dados e atuar sobre os recursos da cidade.
Cidadãos e agentes do governo poderão requisitar esses serviços em diferentes horas do dia.
Desta forma, uma plataforma de Cidades Inteligentes deve estar habilitada para atender toda essa demanda.

A realização de experimentos na escala de grandes cidades se apresenta como um meio de validar o funcionamento de plataformas de Cidades Inteligentes em condições reais, todavia, a realização de
experimentos em tais condições não é uma tarefa fácil.
Experimentos que fazem uso apenas de \textit{scripts} para geração de carga de trabalho sintética (usando ferramentas como \textit{ApacheBench}\footnote{https://httpd.apache.org/docs/2.4/programs/ab.html})
não exercitam as plataformas em condições reais de uma cidade.
Conforme citado anteriormente, praticamente tudo no contexto de uma cidade está conectado.
%Portanto, um dado que seja enviado para a plataforma provelmente gerará uma reação que não terá próposito nesse caso.
Apenas enviar dados para a plataforma sem prover um meio de reação interativo não é o suficiente.
Por exemplo, ao receber um determinado dado em uma certa circunstância, a plataforma pode ter que atuar de alguma maneira na cidade, o que alterará o seu estado, modificando o próximo dado a ser enviado 
para a plataforma.
Portanto, esse tipo de experimento é válido, mas não representa uma situação real do cotidiano de uma cidade.

%Para solucionar essa falta de interação apresentada na solução anterior, podemos fazer uso de \textit{testbeds} na realização dos experimentos.
Para solucionar a falta de interação na utilização de \textit{scripts} para geração de carga de trabalho sintética, podemos fazer uso de \textit{testbeds} na realização dos experimentos.
Esses por sua vez, são plataformas para execução de testes onde no contexto de IoT, em geral, são implantações de dispositivos reais.
O projeto \textit{SmartSantander}~\cite{sanchez_2014} possui um dos \textit{testbeds} mais famosos na área de Cidades Inteligentes.
Vários sensores e atuadores foram implantados na cidade de Santander na Espanha e, com isso, eles se tornaram capazes de realizar experimentos mais realistas.
Entretanto, construir \textit{testbeds} como esse, na escala de uma cidade (mesmo sendo uma cidade pequena), não é trivial.
Existem procedimentos, envolvendo governos e um alto custo associado a compra, instalação e manutenção de toda a infraestrutura necessária.
Sendo assim, construir \textit{testbeds} reais na escala de grandes metrópoles do mundo não é uma solução factível na maioria das vezes.

Nesse sentido, apresentamos a simulação como uma saída viável para o problema.
Podemos simular cidades inteiras, implementando modelos realistas capazes de permitir a interação em tempo real com plataformas de Cidades Inteligentes.
Os diversos recursos da cidade e dispositivos de IoT acoplados aos mesmos são simulados, e uma interface de comunicação entre o simulador e a plataforma pode tornar a comunicação transparente, ou seja,
a plataforma não tem conhecimento se o ambiente é simulado ou não.
%Os diversos recursos da cidade e dispositivos de IoT acoplados aos mesmos, são simulados, através de uma interface de comunicação entre o simulador e a plataforma, a interação da plataforma com esses
%recursos e dispositivos se dá de maneira transparente,.
Ressaltamos que caso o simulador possa executar na escala de grandes cidades, poderá realizar experimentos em cenários realistas.
Acrescentamos o fato de nos tornamos capazes de realizar experimentos envolvendo tecnologias que ainda não são amplamente adotas ou ainda estão em fase de projeto.

Ao fazer uso de um ambiente simulado para realizar experimentos, possibilitamos que os mesmos sejam reprodutíveis.
Utilizar \textit{scripts} para geração de carga também oportunizará reproduzir experimentos (caso os mesmos estejam disponíveis).
Porém, ao usar \textit{testbeds} reais contendo dispositovs de IoT, essa característica não poderá ser garantida, haja vista que, ambientes externos podem apresentar grande variabilidade independentemente da
execucação de um protocolo (por exemplo, elementos climáticos).
A utilização de simulação para resolver problemas similares aos apresentados vem sendo adotada em diversos trabalhos~\cite{karnouskos_2009}~\cite{fleischer_1994}~\cite{dupuy_1990}~\cite{boukerche_2001}.

Com o intuito de viabilizar experimentos com plataformas de Cidades Inteligentes, apresentamos nesta dissertação uma proposta de solução para criação de um ambiente simulado de experimentação.
Para a construção desse ambiente fizemos uso de um simulador e de uma plataforma de Cidades Inteligentes, onde ambos são integrados de acordo com o cenário desejado.
Para tanto, realizamos a implementação dessa solução utilizando o simulador InterSCSimulator~\cite{santana_17} e a plataforma InterSCity~\cite{arthur_17} atraveś de dois cenários de Cidades Inteligentes
apresentados como estudo de caso e descritos posteriormente.

\section{Motivação}

O elemento motivador para a execução deste trabalho foi a busca pela melhoria do processo de experimentação de plataformas de Cidades Inteligentes.
No decorrer das discussões no nosso grupo de pesquisa, percebemos que realizar experimentos em cenários que se assemelham à realidade ainda constitui-se um desafio.
Experimentos realistas são essenciais para validarmos o bom funcionamento dessas plataformas em ambientes de produção.

Conforme dito anteriormente, existem alguns mecanismos que já vêm sendo utilizados na realização de experimentos com plataformas de Cidades Inteligentes, contudo, não proporcionam o nível de interação
e a escalabilidade necessária.
A solução apresentada neste trabalho reduz os custos quando comparamos a \textit{testbeds} reais, e ainda nos permite investigar cenários hipotéticos, com tecnologias futuristas ou ainda não
passíveis de implantação.

\section{Objetivo e Contribuições}

O objetivo central desta pesquisa é prover um mecanismo para construção de um ambiente de experimentação de larga escala e interativo para plataformas de Cidades Inteligentes através de simulação.
Para tanto, uma proposta de solução foi projetada visando atender os principais requisitos de um ambiente com essa característica.
Realizamos também a implementação de dois cenários de Cidades Inteligentes seguindo a arquitetura da solução proposta, usando o simulador InterSCSimulator e a plataforma InterSCity.
Tais implementações foram usadas para a realização de dois estudos de casos propostos para validar a solução.

Durante a realização deste estudo, diversas contribuições foram realizadas tanto para o InterSCSimulator quanto para a InterSCity.
Foi possível adicionar novos cenários ao simulador, melhorias nos cenários existentes, além de mecanismos de cache adicionados à plataforma.
Ademais, seguindo a ideia de reprodutibilidade e infraestrutura como código\footnote{https://en.wikipedia.org/wiki/Infrastructure\_as\_code} (\textit{Infrastructure As Code} -- IaC), automatizamos e
disponibilizamos todos os \textit{scripts}, arquivos de configuração e documentação necessária para a reprodução dos experimentos realizados neste trabalho.


\section{Organização do Trabalho}

No Capítulo \ref{cap:trab-relacionados} serão apresentados os principais trabalhos relacionados e conceitos envolvidos nesta pesquisa.
Uma discussão acerca da proposta de solução é apontada no Capítulo \ref{cap:proposta}.
O Capítulo \ref{cap:estudos-de-caso} disserta sobre os dois estudos de caso realizados, incluindo as suas implementanções e ferramentas utilizadas.
Por fim, as conclusões e possibilidades de trabalhos futuros serão apresentadas no Capítulo \ref{cap:conclusao}.
