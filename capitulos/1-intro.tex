\chapter{Introdução}
\label{cap:introducao}

% Contextualização: cidades inteligentes

A utilização de Tecnologia da Informação e Comunicação - \textit{TIC} - para solucionar os problemas das cidades está surgindo como uma boa estratégia para mitigar os problemas que são agravados pelo rápido crescimento da população urbana~\cite{chourabi_2012}.
Essa abordagem visa otimizar os serviços e uso de recursos da cidade, assim como melhorar a qualidade de vida dos cidadãos~\cite{santana_2016}.

Para que plataformas de Cidades Inteligentes atendam as expectativas de governos e cidadãos na melhoria da qualidade de vida nas cidades, elas precisam ser capazes de lidar com todas as adversidades e
nuância presentes nas cidades modernas.
Devem ser flexíveis e capazes de escalar para atender as diferentes demandas no decorrer do dia, além de estar apta a trabalhar com múltiplos aspectos que tornam o contexto de Cidades Inteligentes tão
peculiar.
Nas cidades, tudo é conectado, onde um aspecto influencia e é influenciado por diversos outros.

Em uma grande metrópole como São Paulo, milhões de recursos (ônibus, carros, hospitais, estacionamentos, escolas, etc.) deverão ser monitorados e gerenciados em tempo real;
dispositivos de Internet das Coisas (\textit{Internet of Things - IoT}), com o objetivo de obter dados e atuar sobre os recursos da cidade, poderão enviar e receber dados a todo momento concorrentemente;
cidadãos e agentes do governo poderão requisitar esses serviços em diferentes horas do dia.
Portanto, uma plataforma de Cidades Inteligentes deve estar habilitada para atender toda essa demanda.

A realização de experimentos na escala de grandes cidades se apresenta como um meio de validar o funcionamento de plataformas de Cidades Inteligentes em condições reais.
Todavia, a realização de experimentos em tais condições não é uma tarefa fácil.

Experimentos que fazem uso apenas de \textit{scripts} para geração de carga de trabalho sintética (usando ferramentas como \textit{ApacheBench}\footnote{https://httpd.apache.org/docs/2.4/programs/ab.html})
não exercitam as plataformas em condições reais de uma cidade.
Como foi mencionado anteriormente, praticamente tudo no contexto de uma cidade está conectado.
Portanto, um dado que seja enviado para a plataforma provelmente gerará uma reação que não terá próposito nesse caso.
Por exemplo, ao receber um determinado dado em uma dada circunstância a plataforma pode ter que atuar de alguma forma na cidade, o que alterará o estado da cidade, modificando o próximo dado a ser enviado 
para a plataforma.
Portanto, esse tipo de experimento é valido, mas não representa uma situação real do cotidiano de uma cidade.

Para solucionar essa falta de interação apresentada na solução anterior, podemos fazer uso de \textit{testbeds} na realização dos experimentos.
\textit{Testbeds} são plataformas para execução de testes, onde no contexto de IoT, em geral, são implantações de dispositivos reais.
O projeto \textit{SmartSantander}~\cite{sanchez_2014} possui um dos \textit{testbeds} mais famosos na área de Cidades Inteligentes.
Vários sensores e atuadores foram implantados na cidade de Santander na Espanha, e com isso eles se tornaram capazes de realizar experimentos mais realistas.
Entretanto, construir \textit{testbeds} como esse, na escala de uma cidade (mesmo sendo uma cidade pequena), não é trivial.
Existem burocracias envolvendo governos e um alto custo associado a compra, instalação e manutenção de toda a infraestrutura necessária.
Sendo assim, construir \textit{testbeds} reais na escala de grandes metrópoles do mundo não é uma solução factível na maioria das vezes.

Nesse sentido, apresentamos a simulação como uma saída viável para o problema.
Com ela podemos simular cidades inteiras, implementado modelos realistas, e permitindo a interação em tempo real com plataformas de Cidades Inteligentes.
Os diversos recursos da cidade e dispositivos de IoT acoplados aos mesmos são simulados, e uma interface de comunicação entre o simulador e a plataforma pode tornar a comunicação transparente.
Com isso, sendo o simulador capaz de executar na escala de grandes cidades, seremos capazes de realizar experimentos em cenários realistas.
Além do mais, nos tornamos capazes de realizar experimentos envolvendo tecnologias que ainda não são amplamente adotas ou não foram de implementadas.

Ao fazer uso de um ambiente simulado para realizar experimentos, fazemos com que os mesmos sejam reprodutíveis.
Ao utilizar \textit{scripts} para geração de carga também somos capaz de reproduzir experimentos (caso o mesmo esteja disponível), contudo, ao usar \textit{testbeds} reais contendo dispositovs de IoT
essa característica não é mais garantida.
A utilização de simulação para resolver problemas similares aos apresentados vem sendo adotada em diversos trabalhos~\cite{karnouskos_2009}~\cite{fleischer_1994}~\cite{dupuy_1990}~\cite{boukerche_2001}.

Com o intuito de viabilizar experimentos de plataformas de Cidades Inteligentes, apresentaremos uma proposta de solução para criação de um ambiente simulado de experimentação.
Para criação desse ambiente fazemos uso de um simulador e uma plataforma de Cidades Inteligentes, onde ambos são integrados de acordo com o cenário desejado.
Além disso, realizamos a implementação dessa solução utilizando o simulador InterSCSimulator~\cite{santana_17} e a plataforma InterSCity~\cite{arthur_17} atraveś de dois cenários de Cidades Inteligentes
que serão apresentados adiante.
E por fim, realizamos dois experimentos, um envolvendo cada cenário, para demonstrar a funcionalidade do ambiente construído e validar a implementação feita.

\section{Motivação}

O principal estímulo para a execução deste trabalho é melhorar e facilitar o processo de experimentação de plataformas de Cidades Inteligentes.
No decorrer de discussões dentro do nosso grupo de pesquisa, percebemos que realizar experimentos em cenários que se assemelham a realidade ainda é um desafio, e sendo esses essenciais para validarmos o bom
funcionamento dessas plataformas em ambientes de produção.

Como dito anteriormente, existem alguns mecanismos que já vêm sendo utilizados na realização de experimentos com plataformas de Cidades Inteligentes, contudo, os mesmos não proporcionam o nível de interação
e a escalabilidade necessária.
Além do mais, a solução apresentada neste trabalho reduz os custos quando comparamos a \textit{testbeds} reais, e ainda nos permite investigar cenários hipotéticos, com tecnologias futuristas ou ainda não
passíveis de implantação.
Portanto, vimos a oportunidade de desenvolver de um trabalho precursor na área.

\section{Objetivo e Contribuições}

O objetivo central deste trabalho é prover um mecanismo para construção de um ambiente de experimentação de larga escala e interativo para plataformas de Cidades Inteligentes através de simulação.
Para isso, uma proposta de solução foi projetada visando atender os principais requisitos de um ambiente como esse.
Além disso, realizamos a implementação de dois cenários de Cidades Inteligentes seguindo a arquitetura da solução proposta, usando o simulador InterSCSimulator e a plataforma InterSCity.
Tais implementações foram usadas para a realização de dois estudos de casos que serão apresentados no decorrer do texto, com o intuito de validar a proposta de solução e implementação feita.

Durante a realização desta pesquisa, diversas contribuições a nível de código foram realizadas tanto para o InterSCSimulator quanto para a InterSCity.
Novos cenários de simulação forma implementados no InterSCSimulator, além de melhorias feitas nos cenários existentes, e mecanismos de cache foram adicionados a plataforma InterSCity.
Ademais, seguindo a ideia de reprodutibilidade e infraestrutura como código\footnote{https://en.wikipedia.org/wiki/Infrastructure\_as\_code} (\textit{Infrastructure As Code} -- IaC), automatizamos e
disponibilizamos todos os \textit{scripts}, arquivos de configuração e documentação necessária para a reprodução dos experimentos realizados neste trabalho.


\section{Organização do Trabalho}

No Capítulo \ref{cap:trab-relacionados} serão apresentados os principais trabalhos relacionados e conceitos envolvidos nesta pesquisa.
Uma discussão acerca da proposta de solução bem como os exemplos de implementação serão apresentados no Capítulo \ref{cap:proposta}.
O Capítulo \ref{cap:estudos-de-caso} apresenta os dois estudos de caso realizados, fazendo uso das implementações do capítulo anterior.
Por fim, as conclusões e trabalhos futuros serão apresentadas no Capítulo \ref{cap:conclusao}.
