\chapter{Proposta de Solução}
\label{cap:proposta}

A solucao proposta para permitir a realizacao de experimentos de plataformas de Cidades Inteligentes em ambientes emulados abrangendo diferentes cenarios em diferentes escalas sera
apresentada neste capitulo.
A discussao sobre os requisitos de uma possivel solucao para esse problema se dara de maneira generica, contudo, ao abordarmos a arquitetura, entraremos em alguns detalhes referentes ao
simulador e emulador de Cidades Inteligentes InterSCSimulator, a ferramenta utilizada neste trabalho.
Alem disso, serao apresentados alguns exemplos de implementacao de cenarios de Cidades Inteligentes usando o InterSCSimulator seguindo a arquitetura proposta, trazendo em seguida os
principais problemas e dificuldades encontradas nesse processo.

\section{Requisitos da Solução}

Para a realizacao de experimentos em ambientes emulados de Cidades Inteligentes temos requisitos em diferentes niveis, desde a capacidade do emulador de representar o contexto
de uma cidade ate formas de comunicacao que permita a interacao entre o emulador e a plataforma de Cidades Inteligentes.
Portanto, dividimos os requisitos em requisitos de \textbf{emulacao} e de \textbf{integracao}.

Para que uma ferramenta possa refletir o contexto de Cidades Inteligentes, onde o sensoriamento de diversos aspectos das cidades ja e uma realidade, e a atuacao (modificacao do estado
das cidades) segue uma crescente, consideramos que a mesma deva de alguma forma ter os conceitos abaixo representados:

\begin{itemize}
    \item \textit{Recursos da cidade}: esse recursos da cidade, sendo eles qualquer coisa no contexto da cidade (automoveis, aparelhos publicos, vias e etc.), serao objetos tanto de
        sensoriamento e/ou atuacao por parte das plataformas de Cidades Inteligentes.

    \item \textit{Modelos mais realistas possiveis}: a ferramenta deve implementar modelos que representem ao maximo a realidade de uma cidade, e nao apenas casos hipoteticos.
        Por exemplo, modelos de fluxo de automoveis nas vias devem incluir possiveis engarrafamentos.

    \item \textit{Capacidades inerentes a cada recurso da cidade}: cada \textit{recurso da cidade} tem as suas respectivas capacidades, sendo elas de sensoriamento ou atuacao.
        Por exemplo, vagas de estacionamento podem ser monitoradas quanto a sua ocupacao, e semaforos de transito podem ter o seu estado modificado atraves de atuacao.

    \item \textit{Tempo real}: a dinamica da cidade deve ser emulada, ou seja, em tempo real. Essa dinamica representada nao deve ser executada nem mais rapida nem mais lenta do que
        aconteceria em um ambiente real de uma cidade.
        Esse e um requisito importante pois o desencadeamento de acoes nas cidades sao muitas vezes devido ao processamento de acoes acontecidas anteriormente, e as plataformas
        devem ter tempo habil para realizar tais acoes, como foi explicado nos capitulos anteriores.
\end{itemize}

Consideramos os pontos apresentados acima como \textbf{requisitos de emulacao}, sendo esses o minimo necessario para conseguirmos de fato representar um ambiente real de uma cidade.
Caso a ferramenta atenda os requisitos apresentados, se torna possivel a emulacao de cenarios realistas de uma cidade contendo todos os recursos necessarios, onde os mesmos podem ser
sensoriados e/ou sofrerem atuacoes, sendo esse o contexto esperado pra um experimento de uma plataforma de Cidades Inteligentes.
Todavia, apenas essa emulacao nao viabiliza ainda a realizacao de experimentos com as plataformas, ainda precisamos atender alguns \textbf{requisitos de integracao} entre o emulador
e a plataforma em questao.

Para termos um ambiente integrado precisamos definir um meio de comunicacao de duas vias, envio e recebimento de dados, em tempo real entre o emulador e a plataforma de Cidades Inteligentes.
Abaixo sao apresentados os \textbf{requisitos de integracao} necessarios para obtermos um ambiente integrado de experimentacao para essas plataformas:

\begin{itemize}
    \item \textit{Envio de dados de sensoriamento em tempo real}: para que as funcionalidades relativas a coleta e armazenamento de dados, e processamento dos mesmos possam ser
        exercitadas pelas plataformas, precisamos enviar dados de todos os recursos presentes no cenario de experimentacao com tais capacidades em tempo real.
        Quanto maior for a escala do cenario de experimentacao mais dificil se torna essa tarefa, pois se faz necessario enviar milhoes de dados de sensores simultaneamente atraves
        do mesmo canal.
        Apesar de a demanda de tempo real poder se tornar um problema, ela se faz necessaria para que possamos chegar mais proximo de um cenario realista, onde recursos sensoriados
        da cidade poderao nao aguardar outros enviarem seus dados para enviar, apenas enviarao no tempo estipulado.

    \item \textit{Recebimento de dados de atucao em tempo real}: as plataformas, em dados cenarios de experimentacao, precisarao enviar comandos de atuacao para a cidade (neste caso,
        para o ambiente emulado) em tempo real.
        Diferente dos dados de sensoriamento, o fluxo de dados de atuacao e bem menor.
        Contudo, esse tipo de dado tem a sua propria peculiaridade, pois o emulador deve ser capaz de consumir essa dado em tempo de execucao e alterar o estado do ambiente emulado
        imediatamente.
        Isso porque ao modificar de alguma forma o estado do ambiente emulado influenciamos diretamente os dados de sensores naquele dado momento em diante, e os mesmos continuam
        sendo enviados para a plataforma em paralelo.
\end{itemize}

Portanto, acreditamos que ao termos um ambiente que atenda tanto os \textbf{requisitos de emulacao} quanto os de \textbf{integracao}, se torna possivel realizarmos experimentos
com plataformas de Cidades Inteligentes sem a necessidade de \textit{testbeds} reais.
Isso nos possibilita testar cenarios em escalas reais de uma cidade, alem de nos permitir emular cenarios que ainda nao sao passiveis de reproducao atraves de \textit{testbeds},
emulando tecnologias ainda nao existentes e cenarios futuristas.
Por exemplo, podemos emular um ambiente onde todos os carros da cidade sao autonomos se comunicando atraves de tecnologia \textit{5G}.

\section{Arquitetura}
\section{Exemplos}
\section{Problemas e Dificuldades}
