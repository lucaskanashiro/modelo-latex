\chapter{Revisão Bibliográfica e Trabalhos Relacionados}
\label{cap:trab-relacionados}

Durante a realização desta pesquisa, não foi encontrada uma solução que provesse um ambiente de experimentação para plataformas de Cidades Inteligentes que permitisse a interação em tempo de execução 
(viabilizando a execução de comandos de atuação, por exemplo) em larga escala, sendo esses requisitos ainda pouco explorados pela academia.
Para um melhor entendimento de como esses experimentos estão sendo realizados, serão introduzidos trabalhos relacionados que apresentam ambientes de experimentação nessa área, apontando as suas
principais limitações.
Alguns trabalhos apontam a simulação como um método que pode solucionar algumas das limitações desses ambientes, esses também serão discutidos e explicitado o porquê de adotarmos tal solução.

Além disso, existem pesquisas que já se utilizam da integração com simuladores para a execução de experimentos que se assemelham a contextos reais.
Nesse sentido, apresentaremos as principais dificuldades na integração de simuladores e sistemas de software em geral.


\section{Ambientes de Experimentação para Plataformas de Cidades Inteligentes}

%Apresentar e discutir testbeds existentes

Apesar dos significativos avanços tecnológicos, dificuldades associadas com a avaliação de plataformas para Cidades Inteligentes sob condições realísticas em ambientes experimentais ainda dificultam
a sua maturidade e utilização significativa~\cite{sanchez_2014}.
Por isso, encontrar soluções eficazes para melhorar e facilitar a criação de ambientes de experimentação ainda é um problema não resolvido.

No contexto de Cidades Inteligentes, onde em grande parte interagimos com dispositivos de IoT (\textit{Internet of Things}), temos alguns desafios que dificultam a criação desses ambientes de
experimentação, sendo os pricipais \textit{escala}, \textit{heterogeneidade de dispositivos}, \textit{acesso concorrente}, \textit{mobilidade} e 
\textit{reprodutibilidade}~\cite{gluhak_2011}~\cite{sanchez_2014}.
O primeiro deles é a \textit{escala} desses ambientes, tendo em vista que as cidades e suas populações estão em uma constante crescente, os experimentos envolvendos essas plataformas devem ser capazes
de refletir isso.
Na infraestrutura física de uma grande cidade, como São Paulo, teremos milhares de dispositivos de IoT de diferentes naturezas monitorando e atuando na cidade, e os experimentos para que sejam realistas
precisam representar essa \textit{heterogeneidade de dispositivos}.
Além disso, em uma Cidade Inteligente teremos diversas aplicações tentando acessar recursos da infraestrutura da cidade simultaneamente, esse \textit{acesso concorrente} deve também estar presente nos
ambientes de experimentação.
Todos esses desafios anteriores dificultam ainda mais a \textit{reprodutibilidade} desse ambiente, o que é necessário para reprodução dos experimentos. 

Podemos encontrar diversos trabalhos que tentaram solucionar os desafios citados, sendo grande parte dessas soluções através de \textit{testbeds} reais.
Nesses \textit{testbeds}, dispositivos de IoT são implantados no ambiente, conectados através de redes de sensores e monitorados.
Todo esse processo tem um custo alto, tanto de aquisição dos equipamentos necessários quanto de mão de obra para execução.
Temos \textit{testbeds} implantados em ambientes abertos e também em laboratórios.

Grande parte dos \textit{testbeds} apresentados na literatura foram implantados em cidades da Europa, e cada um deles tem um foco específico.
Os \textit{testbeds} apresentados em  \cite{olivares_2013} e \cite{cenedese_2014} são voltados para experimentos envolvendo redes de sensores, como os nós interagem e o correto
funcionamento de todas as camadas de protocolos.
Ambos tratam a questão da \textit{heterogeneidade de dispositivos} e \textit{mobilidade}, contudo não mencionam questões de \textit{escala} do ambiente,
\textit{acesso concorrente} e \textit{reprodutibilidade}.
Encontramos o \textit{FIT IoT-LAB}~\cite{adjih_2015}, que também é um ambiente de testes focado em redes de sensores como os demais, só que os autores o apresenta como um ambiente de larga escala,
contudo, o mesmo possui 2728 nós de baixo consumo conectados à uma rede sem fio e 117 robôs móveis.
Levando em consideração que em uma cidade como São Paulo esperamos ter dispositivos na escala de milhões, não acreditamos que essa escala apresentada seja suficiente para a realização de experimentos
realistas no contexto de uma grande metrópole.

Ambientes de experimentação voltados especialmente para a \textit{heterogeneidade de dispositivos} são apresentados em \cite{latre_2016} e \cite{juraschek_2012} na forma de \textit{testbeds} localizados
em Antwerp na Bélgica, e Berlim na Alemanha, respectivamente.
Nesses trabalhos, a heterogeneidade é tratada a nível de protocolos de rede para comunicação e tratamento de dados, considerando que cada dispositivo pode utilizar diferentes protocolos e se comunicar
através de diferentes formatos de dados.
Mais uma vez temos o problema de \textit{escala} que não é deixado claro nas apresentações dos \textit{testbeds}.

Em \cite{lanza_2015}, é apresentado uma parte do \textit{testbed} implantado na cidade de Santander na Espanha com uma característica muito interessante que não é encontrada na maioria dos outros
trabalhos, o seu foco principal é a \textit{mobilidade}.
Ao perceberem que veículos são a melhor forma de obter dados da cidade reduzindo os gastos, já que circulam por vários pontos da cidade no mesmo dia, implantaram sensores em 140 ônibus, táxis e vans
para coleta de diversos dados.
Ao coletarem dados de diferentes naturezas, utilizaram diferentes dispositivos, trabalhando também a questão da \textit{heterogeneidade}.
Todavia, todos os desafios foram atacados em uma \textit{escala} bem menor do que a esperada para uma grande cidade.

No contexto de Cidades Inteligentes, além da parte técnica, temos também que nos preocupar com o impacto dessa tecnologia na vida das pessoas.
O \textit{testbed} introduzido por \cite{nati_2013}, tem o objetivo de viabilizar experimentos centrados no usuário, onde ao invés de ter os dispositivos e softwares envolvidos sob prova, temos
o usuário como alvo do experimento.
Nesse tipo de ambiente, seria bastante complexo a realização de experimentos em larga \textit{escala}, tendo em vista que necessitaríamos de envolver a população de uma cidade no processo de
experimentação.
Por isso, esse \textit{testbed} foi implantado no contexto de um escritório, e não ataca a maioria dos desafios citados nesta seção, como \textit{mobilidade} por exemplo.

Temos também uma série de \textit{testbeds} voltados especificamente para apenas uma das áreas de Cidades Inteligentes.
Por exemplo, \cite{braem_2016} foca na análise da qualidade do ar, \cite{lee_2015} apresenta um \textit{testbed} meteorológico, \cite{lu_2010} introduz um ambiente de experimentação envolvendo
\textit{Smart Grid}, e \cite{amrutur_2017} demontra um \textit{testbed} de postes de iluminação pública.

Existem também algumas iniciativas que visam realizar a federação de \textit{testbeds} em diferente localidades.
Em \cite{mwangama_2013} e \cite{corici_2014}, visando balancear a carga de trabalho de ambientes de experimentação em Berlim na Alemanha e na Cidade do Cabo na África do Sul, ambos foram federados,
tratando de pontos como autenticação, identificação segura e políticas de armazenamento e de redirecionamento.

Além de todos esses ambientes de experimentação apresentados, temos provavelmente o mais conhecido dentre eles, o desenvolvido pelo projeto \textit{SmartSantander}~\cite{sanchez_2014}.
Esse \textit{testbed} foi implantado na cidade de Santander na Espanha, e o mesmo visou atacar todos os desafios apresentados nesta seção.
Além de ser um ambiente de teste, ele provê serviços para a população da cidade através de aplicações de Cidades Inteligentes, como por exemplo uma aplicação de estacionamento inteligente.
Ademais, a implantação desse \textit{testbed} teve total apoio do governo da cidade e financiamento de diversas fontes.
Todavia, apesar desse ser um dos ambientes de experimentação mais bem sucedidos que temos na literatura, ainda acreditamos que o mesmo não representa a escala de uma grande metrópole como
São Paulo, sabendo que Santander possui aproximandamente $ 35 km^2 $ e 175 mil habitantes\footnote{https://pt.wikipedia.org/wiki/Santander\_(Cant\%C3\%A1bria)}.

Uma solução diferente da implantação de \textit{testbeds} como ambiente de experimentação no contexto de plataformas de Cidades Inteligentes foi apresentada em \cite{ali_2015}.
O \textit{CityBench} é um \textit{benchmark} para motores de processamento de fluxo de dados (\textit{streaming}) presentes nesse tipo de plataforma.
Esse \textit{becnhmark} foi construído através do fluxo de dados em tempo real de múltiplos sensores implantados na cidade de Aarhus na Dinamarca.
Essa é uma abordagem interessante, onde seria possível coletar dados em \textit{escalas} maiores, adicionar dados referentes a \textit{mobilidade}, de dispositivos \textit{heterogeneos}, permitindo
o \textit{acesso concorrente} e a \textit{reprodutibilidade}.
Contudo, esse tipo de abordagem não permite que a plataforma interaja com o ambiente de experimentação em tempo de execução, não permitindo experimentos envolvendo comandos de atuação por exemplo.
Além do mais, para a criação desse conjunto de dados para a contrução do \textit{benchmark} é necessário ter acesso a dispositivos reais implantados em \textit{testbeds} reais.

Como pode ser visto, nenhum dos trabalhos apresentados provê um ambiente de experimentação para plataforma de Cidades Inteligentes que permita a interação em tempo de execução, possibilitando
atuação durante o experimento.
Além disso, as solução apresentadas não são capazes de realizar experimentos na escala de grande metrópoles do mundo, como a cidade de São Paulo.
Sendo assim, identificamos o potencial deste trabalho, onde visamos solucionar esses dois problemas apresentados e os desafios propostos para contrução desse ambiente de experimentação através
de simulação.

\section{Integração de Ferramentas para prover Ambiente de Experimentação}

%Exemplos de integração de simuladores, como o SUMO (simulador de tráfego) + ns2
%(simulador de rede)


